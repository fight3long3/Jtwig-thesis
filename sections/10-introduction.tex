-\chapter{Introduction}

Sir Tim Berners-Lee published the proposal for the World Wide Web, commonly referred as \textbf{Web}, in 1989 \cite{Berners-Lee1989InformationProposal}. At that time Tim was not predicting how spread and important his creation would be. Although initially rejected, his proposal laid three fundamental building blocks for the web today:

\begin{itemize}
    \item HTTP: Hypertext Transfer Protocol. Allows for the retrieval of linked resources from across the web.
    \item URI: Uniform Resource Identifier, also commonly called a URL, allows to uniquely identity a resource on the entire web.
    \item HTML: HyperText Markup Language. The markup (formatting) language for the web, allowing to write, the so called, \textit{web pages}.
\end{itemize}

Tim also wrote the \textit{WorldWideWeb}, an HTML editor and visualisation tool \cite{Gillies2000HowWeb}, which is known to be the first ever built Browser for the web.

Since then web has evolved, it spread across the globe and is being used as a fundamental resource for communication nowadays. The usage of HTML specifically also evolved, it started with simple static content and became a dynamic, customizable and personalizable way of reaching out to people.

At the same time this evolution was taking place, existing concepts were being applied to this new world. An example of that is the concept of templates.

\section{Template}

However only recently applied in many scientific areas, the concept of template is not that recent, in fact it can tracked back early in the \texttt{XIX} century.

\subsection{Meaning \& Etymology}

In \textit{The Century Dictionary} \cite[p.6224]{Whitney1906TheDictionary}, \texttt{template} entry is just pointing to the \texttt{templet} definition, which is used as the main entry for this concept. Below, there is an excerpt taken from page 6224 of this book.

\begin{bookQuote}
     \textbf{Templet} 
     
     \begin{enumerate}
         \item A pattern, guide, or model used to indicate the shape any piece of work is to assume when finished. It may also be used as a tool in modelling plastic material, or as a guide placed in a milling-machine, shaper-lathe or other automatic cutting-machine. In these applications it maybe a thin piece of wood or metal, with one or all the edges cut in profile to the shape of the baluster, cornice, part of a machine, or other object to be wrought to shape. Templets are also used as guides in filing sheet-metal to shape, as in making small brass gears for clocks, sheets of brass being clamped between steel templets, and all the parts projecting beyond the edges being flied away. Templets are used in founding as patterns in forming moulds in loam.
         
         \item A strip of metal used in boiler-making, pierced with a series of holes, and serving as a guide in marking out a line of rivet-holes.
         \item In building: (a) A short piece of timber or a large stone placed in a wall to receive the impost of a girder, beam, etc., and distribute its weight. (b) A beam or plate spanning a door or window-space to sustain joists and throw their weight on the piers. (c) One of the wedges in a building-block.
         \item Same as temple
         \item In a brilliant, same as bezel
     \end{enumerate}
\end{bookQuote}

According to \cite{Barnhart1988TheEtymology} \texttt{Template} ending was created in the mid \texttt{XIX} century and was due to the association with plate. 

From here, we can easily see that the concept of template exists long before web was invented, even long before programmable computers were invented \cite{Clements2014ComputerVariations}, however, the intent for optimisation was and still is a major characteristic of this technique.

\subsection{Text Processing}

Back to modern days, when the goal is to generate text using programming languages, some might still remember the character manipulation required in some classic languages. This is becoming increasingly easy with the most recent languages, however, creating and concatenating string constructs are still a painful way of generating large quantities of text \cite{Fowler2003PatternsArchitecture}, this lead to the introduction of templates as a concept in the text processing world.

One observed evolution in programming languages and, at the same time, one of the most basic forms of templates, is the string interpolation mechanism. Perl was one of the first languages to provide native support to it \cite{Wall2000ProgrammingPerl}, below there is an example on how to do it in Perl.

\begin{lstlisting}[language=Perl]
my $time = now();
print "Current time is $time";
\end{lstlisting}

In this example, the string can be divided in two parts, one static part and another dynamic. The dynamic part is the variable \texttt{\$time}, whether the static part is the rest. With this approach is possible to specify variables as part of a string construct. In fact, many recent languages support string interpolation, pretty much in the same way as Perl. 

Generically speaking, templates can be used to take advantage of a text pattern with dynamic constructs and generate instances of text. A template, therefore, can be seen as a function with one parameter - a configuration (commonly referred as \textbf{data model}) - which produces text as output. 

\begin{center}
$Template := Configuration \rightarrow Text$
\end{center}

\begin{orientador}
    Should I detail the formal definition of Template based on \cite{Parr2004EnforcingEngines} or just refer it?
\end{orientador}

\section{Web Development : MVC}

When it comes to web development, more specifically, when developing dynamic web pages, templates can take a key role, specially when applied using an MVC (Model View Controller) pattern.

Model View Controller is one of the most used patterns for web development. It was developed by Trygve Reenskaug for Smalltalk late in the 1970s \cite{Reenskaug1979THING-MODEL-VIEW-EDITORPlanningsystem} and it influenced most UI frameworks since then \cite{Fowler2003PatternsArchitecture}.

It is important to remind the reader that this pattern was drafted before Web was invented therefore the concept required some modifications in order to adapt to the web architecture.

\subsection{Original MVC}

The original Model-View-Controller was built in a time where applications took complete control over the computer, with the ability to take control over both input and output devices. It was created with the main goal of bridging the gap between the user mental model and the digital model maintained by the system. A key component in this pattern is the Model, which represent the knowledge of the system backed by a source of data (database). This is the digital representation of the information on the system. To allow the user to interact with this digital knowledge, two concepts were defined, they are the View and the Controller. The View is a visual representation of the Model, it is responsible of interacting with the model in order to gather data to be shown in the output devices also as update it. The Controller is the bridge between the user input and the system. It is also responsible of organising the views in a proper manner \cite{Reenskaug1979THING-MODEL-VIEW-EDITORPlanningsystem}.


\subsection{Benefits of MVC}

The main benefit of adopting MVC when developing user facing applications it’s the separation of concerns it embodies. The Model and View address two completely distinct domains, while the Model takes care of the business rules, the View addresses the mechanisms of the UI. The Controller is commonly seen as the bridge between View and Model. This approach makes the View dependent on the Model, but that dependency doesn’t exist in reverse, that means, the Model is independent from the View, therefore allowing people developing the Model complete unaware of the View \cite{Parr2008TheSeparation, Parr2004EnforcingEngines, Fowler2003PatternsArchitecture}.

Splitting the business logic domain and display logic allows to have different people with a distinct set of skills addressing each layer individually, making it easier to maintain and think about.

\subsection{Web MVC}

As per Martin Fowler \cite{Fowler2003PatternsArchitecture} definition, the adoption of MVC on the Web suffers a slight adaptation. The Model concept still accounts for the representation of the knowledge, however the View is only about displaying information on the UI. All the changes in the Model are performed by the Controller component. The controller bridges the gap between the user requests and system, also taking care of updating the views accordingly.

\subsubsection{Controller}

There is multiple ways to approach the Controller layer in an MVC pattern within Web Development. The most commonly used are Page Controller and Front Controller. In the first one, each page or action have a specific handler, where in the later there is one handler managing and routing all the incoming requests, usually the Front Controller approach delegates to specific handlers making both Page and Front Controller very similar. 

\subsubsection{Template View}

Template View is an approach used in the display stage of the MVC pattern applied to the Web, the idea behind it is to embed markers in HTML files which are computed and replaced by the result of processing them.

In order to specify a template, a language is required, this is commonly referred as the \textit{Template Language}. Template languages have an inverted approach when compared to common programming languages, that is, programming languages allow developers to embed string definitions (text) in it, like Perl, for example, as shown before, whether, template languages are embedded in the text.

Of course, a template instance on it's own is not enough to produce content. Associated with a template language there is always an engine to process it, so called, \textit{Template Engine}. This engine is an application which applies the template to a specific configuration generating, therefore, text as output.

\begin{center}
    $TemplateEngine := Template \times Configuration \rightarrow Text$
\end{center}

\subsection{Template Engines : Examples}

Wikipedia collected \cite{WikipediaComparisonEngines} until the current date, an incomplete list of, approximately, 100 distinct template engines with associated languages.

\begin{expand}
    Talk about examples of existing template engines:
    
    \begin{itemize}
        \item Perl Template Toolkit \cite{Chamberlain2003PerlToolkit}
        \item Django Templates \cite{Newman2008DjangoMore} 
    \end{itemize}
\end{expand}


\section{MVC in Java Webapps}

Java Webapps are defined by the jsr.. it embodies a set of APIs to enable Java developers to implement web applications. The main concept is the Servlet, an handler for a specific URL pattern. This Servlets require an webapp container able to associate the URL pattern to a specific Servlet implementation, an example of that is the Apache Tomcat. 

MVC, specially using a Front Controller approach, created an opportunity for Frameworks to appear. 

\begin{expand}
    Expand the concept of frameworks \cite{Liu1996SmalltalkDesign}
\end{expand}

In Java, several technologies were created and offered to the general public as open source contributions. Spring Framework and Struts are two well known examples.

This frameworks take advantage of the Front Controller approach and set one handler which then routes the requests to smaller components. This Frameworks ease the initial setup and work of developing web applications providing APIs for different purposes, one of those is the integration with a Template View approach.

\begin{expand}
    Talk about Spring...\cite{Johnson2005ProfessionalFramework}
    
    Spring specifically evolved from a simple Dependency Injection and Inversion of Control with MVC support Framework into a community with distinct projects to address different needs in Java applications.
\end{expand}

When it comes to web development using Java \cite{Bloch2008EffectiveSeries}, more specifically, Java WebApps \cite{Williams2014ProfessionalApplications}.

\begin{expand}
    Talk about examples of template engines for Java, used together with previous frameworks.
    \begin{itemize}
        \item Freemarker \cite{Forsythe2013InstantFreemarker}
        \item Mustache
        \cite{Forsythe2013InstantFreemarker}
    \end{itemize}
\end{expand}

\section{The Problem}

There is already a set of template technologies available in the market for the Java World, however, \claim{existing solutions have some limitations}, more specifically, \claim{long learning curve}, \claim{poor readability}, \claim{poor extendability} and, for some of them, \claim{poor expressing power}. 

\begin{expand}
    Show evidences of previous claims using the previousy mentioned template engines.
\end{expand}

\begin{expand}
    Talk about the consequences of the problems identified.
\end{expand}

\section{The Proposed Solution}

Given this, one decided to come up with a solution that could, potentially, address all of this problems, therefore, creating an alternative with capabilities not yet available in the market. This means:


\begin{orientador}
    How much detail should I give about the solution we are proposing to develop?
\end{orientador}