\chapter{Introduction}

Sir Tim Berners-Lee presented the world wide web, also known as WWW, in 1989. At that time Tim Berners-Lee was not predicting how basic and important his creation would be. Although initially rejected, his proposal laid three fundamental building blocks for the web today:

\begin{itemize}
    \item HTTP: Hypertext Transfer Protocol. Allows for the retrieval of linked resources from across the web.
    \item URI: Uniform Resource Identifier, also commonly called a URL, allows to uniquely identity a resource on the entire web.
    \item HTML: HyperText Markup Language. The markup (formatting) language for the web.
\end{itemize}

He also wrote the \textit{WorldWideWeb}, an HTML editor and visualisation application \reference{Gillies2000HowWeb}, that is the first ever built Browser for the web.

Since then web has evolved, it spread across the globe and is being used as a fundamental resource for communication nowadays. HTML also evolved, it started as static content and became a dynamic, customizable and personalizable way of reaching out to people.

At the same time this evolution was taking place, tools and concepts were being created/applied to this new world. An example of that is the concept of templates. Such concept was brought to the world of the web because of the benefits and possibilities it gives, specially, when it comes to adapt the resulting HTML and generate different flavours of content with a common structure.

Web development using Java \reference{Bloch2008EffectiveSeries}, more specifically, Java WebApps \reference{Williams2014ProfessionalApplications} has HTML formatted content \reference{Duckett2014HTMLWebsites} as the main outcome.

With the era of web personalization and costumization, 
When it comes to produce content with, specially adaptable content with a common structure, templating is always a good resource, it allows to  